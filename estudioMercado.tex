\chapter{Estudio de mercado}
\label{chap:mercado}

\section{Análisis del macroentorno}
\label{sec:macroentorno}

Existen varios elementos del entorno que nos afectan directamente:

\begin{itemize}
\item La situación económica: la sociedad no está pasando por un buen  momento, lo que hace más difícil que éstos decidan pasar fuera de su casa unos días con el fin de no hacer gastos necesarios.
\item La cultura: nos encontramos en una zona en la que la caza está muy arraigada en la sociedad y la cultura vegetariana no es muy compartida con los habitantes de la zona.
\end{itemize}

\section{Análisis del microentorno}
\label{sec:microentorno}
Como elementos del microentorno, los clientes y la competencia son de máxima importancia y se desarrollará de forma exhaustiva en este apartado. También como elementos del microentorno tenemos a los proveedores que van a intervenir directamente en nuestra actividad.

Estamos en contacto con agricultores de la zona que produzcan productos ecológicos para dar a nuestros clientes productos de alta calidad.

\begin{itemize}
\item Proveedores de comida vegetariana
	\begin{itemize}
	\item Soria Natural, SL.
	\item Natursoy S.L
	\end{itemize}
\item Proveedores de bebidas
	\begin{itemize}
	\item Bebidas Cabrera, S.A.
	\item Bebidas alcohólicas Gallo. S.L
	\end{itemize}
\end{itemize}
Los proveedores disponen de certificación ecológica establecida por el Reglamento (CE) 834/2007 el Consejo sobre producción y etiquetado de los productos ecológicos y por el que se deroga en el Reglamento (CEE) 2092/91 y en España por el Reglamento y Normas Técnicas del Consejo Regulador de la Agricultura Ecológica CRAE (1990).

\section{Los clientes}
\label{sec:clientes}

Como he dicho anteriormente este negocio va destinado a un cliente verde o ecológico.

La crisis ecológica que sufre nuestro planeta debe su aparición a un sistema de producción y consumo que exige un nivel de utilización de recursos naturales, de generación de residuos y contaminantes que sobrepasa la capacidad de la naturaleza de autorregenerarse.

Esto ha generado una tendencia mundial al aumento constante del número de empresas que siguen normas sobre la cuestión. Se trata no sólo de elevar el nivel de conciencia de los directivos acerca de la cuestión medioambiental sino de crear toda una posición filosófica acerca de la relación empresa - entorno donde la ética ecológica basada en la preservación del medio ambiente natural no entre en contradicción con los objetivos económicos de la empresa, más aún, que pueda lograrse desde esta posición más eficiencia económica. Por tanto, la respuesta empresarial debe ir más allá de la elaboración de estrategias para satisfacer a los consumidores de productos ecológicos y aprovechar esa oportunidad del mercado, hace falta un concepto global que penetre en todas las áreas y funciones de la empresa y forme parte de su sistema de valores y de la cultura--organizacional. 

De ahí que la empresa deba prestar especial atención a la opinión pública y no sólo a los indicadores económicos, pues la opinión desfavorable de la sociedad podría ocasionar trastornos   en  el  desenvolvimiento empresarial, y  esta conducta empresarial representa un factor decisivo para el posicionamiento competitivo.

La preocupación por el deterioro del medio ambiente no es sólo una  compleja  tendencia social, es también un fenómeno de marketing, el cual está dando lugar a la aparición de un nuevo segmento de consumidores: los consumidores verdes (mencionados anteriormente).

El consumidor verde o ecológico se puede definir como aquel consumidor que manifiesta su preocupación por el medio ambiente en su comportamiento de compra, buscando productos que sean percibidos como de menor impacto sobre el medio ambiente.

Para estos consumidores el calificativo ecológico es un atributo valorado en el proceso de decisión de compra. En algunos casos dicha valoración se manifestará en pagar un mayor precio por productos percibidos como ecológicos; en otros casos se manifestará en el rechazo de aquellos productos más contaminantes; y en otros casos se manifestará en preferir el producto más ecológico en igualdad de condiciones funcionales (calidad, comodidad,…) y económicas (precio, promoción de ventas, cantidad,…).

Respecto al sector de la comida vegetariana, hasta hace algo más de una década el consumo de comida vegetariana no era un hábito extendido. Sin embargo, en los últimos años se ha potenciado el consumo de este tipo de alimentación.

El número de clientes sensibilizados y preocupados por una alimentación sana es cada vez mayor. Conforme pasa el tiempo, los ciudadanos van tomando mayor conciencia de la incidencia de la alimentación diaria en su salud.

Cada día más gente sigue determinadas dietas y quiere controlar lo que come. En realidad se suman tres tendencias:
\begin{itemize}
\item Mayor preocupación por la salud en general. 
\item Mayor conocimiento dietético y conciencia de la importancia de la alimentación en la salud. 
\item Mayor desconfianza en los alimentos convencionales.
\end{itemize}

Las causas de esta tendencia se encuentran en diversos motivos:
\begin{itemize}
\item Acudir a una dieta saludable para combatir enfermedades coronarias, alergias, intolerancia a determinados alimentos… 
\item Mayor concienciación o compromiso ecologista: conservación medioambiental, respeto a la vida de los animales… etc.
\end{itemize}

Estos factores han originado que el número de personas que se autodefinen como vegetarianas vaya en aumento, favoreciendo las posibilidades del sector.

Con todo lo dicho anteriormente, podemos dividir a los clientes de restaurantes vegetarianos en:

\begin{itemize}
\item Clientes comunes: personas que tienen una preocupación media por la calidad de los alimentos que comen. Les parece bien comer sano pero no mantienen una dieta especial ni buscan de forma activa un perfil determinado de alimentos.
\item Clientes sensibilizados: personas especialmente preocupadas porque su alimentación diaria sea sana. Siguen un determinado tipo de dieta o limitación de lo que comen y buscan activamente productos que encajen en ella. Pueden ser de distintos perfiles:
\begin{itemize}
\item Vegetarianos. 
\item Ecologistas. 
\item Deportistas. 
\item Personas con limitaciones por cuestiones médicas (alergias e intolerancias, problemas coronarios, colesterol, ácido úrico,…) Etc.
\end{itemize}
\end{itemize}

Todos ellos tienen en común la demanda de alimentos saludables y estar interesados en saber y controlar lo que comen. Además de comida sana suelen procurar seguridad alimentaria. Les gusta consumir productos fiables, de procedencia conocida y controlada, aspecto que se intensificó en el 2.001 a raíz de la crisis de las vacas locas, fiebre aftosa, peste porcina, etc.

\section{La competencia}
\label{sec:competencia}

Competencia directa en el pueblo no hay ninguna ya que sería el único restaurante del pueblo de este estilo.
%imagen
\begin{figure}[h]
  \begin{center}
    \includegraphics[scale=0.55]{images/competencia.jpg}
    \caption{Ubicación de establecimientos similares en la zona}
    \label{fig:competencia}
  \end{center}
\end{figure}

La mayoría de los establecimientos de la zona (ver Figura \ref{fig:competencia}) ofrecen a sus clientes en sus cartas, platos tradicionales sin ninguna utilización de productos ecológicos. Por lo cual podríamos decir que el restaurante \textbf{\textit{Quercus pyrenaica}}, al ofrecer una oferta gastronómica de productos ecológicos y vegetarianos, no posee una competencia directa, ya que los clientes que van a venir a nuestro establecimiento, son clientes que precisamente se quieren alejar de ese tipo de establecimientos.

