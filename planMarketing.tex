\chapter{Plan de marketing}
\label{chap:marketing}

Los componentes del marketing son: el producto y el servicio y el precio. 

\section{Descripción del producto y del servicio}
\label{sec:desProduct}

El negocio consiste principalmente en dar alojamiento rural a esta creciente demanda de clientes verdes y hacer que disfruten de una comida sana y ecológica, basada principalmente en productos de temporada o tratados de la mejor forma posible, para que lleguen a la mesa del comensal con la mayor calidad posible. Para ello, se utilizarán técnicas de conservación como son: 

\begin{itemize}
\item Refrigeración: sometiendo al producto a baja temperaturas superiores a 0ºC.
\item Congelación: sometiendo a los productos a temperaturas inferiores a los -18ºC.
\item Envasado al vacío: ausencia de atmosfera.
\item Deshidratación: cosiste en extraer parte de la humedad de los géneros tratados. Esta técnica la realizaremos de la forma más natural posible pero con ayuda de deshidratadores solares que cumplan con la normativa.

%imagen
\begin{figure}[h]
  \begin{center}
    \includegraphics[scale=0.25]{images/deshidratador1.jpg}
    \includegraphics[scale=0.81]{images/deshidratador2.jpg}
    \caption{Deshidratador artesanal}
    \label{fig:deshidratador}
  \end{center}
\end{figure}

\item Encurtidos: utilizando vinagre como método de conservación aumentando la acidez del medio para la proliferación de bacterias.
\item Adobos: este método actúa por medio de ingredientes conservadores. No es utilizado tanto como método de conservación, si no por su sabor, ya que algunos platos de la carta están caracterizados por el mismo.
\item Esterilización: método por el calor para destruir bacterias y microorganismos.
\item Utilización de atmósferas modificadas 
\end{itemize}

Algunos de los productos elaborados que se comercializan en el establecimiento se dividen en:

\begin{itemize}
\item Variedad de cereales, legumbres, semillas y frutos secos. 
\item Verduras, algas marinas y frutas de la temporada (siempre que sea posible). 
\item Aceite de oliva.
\item Bebidas: infusiones, café, zumos y licuados, cervezas, vinos, licores.
\end{itemize}

Aparte de las funciones convencionales del restaurante mencionadas anteriormente se ofrece:

\begin{itemize}
\item Alojamiento en 6 bungalows con capacidad de dos personas y 2 de cuatro personas
\item Cursos de formación
\begin{itemize}
\item Curso de recolección de setas de temporada.
\item Curso de agricultura ecológica
\item Curso de jardinería (mantenimiento, podas y composiciones florales)
\item Curso de cocina vegetariana.
\end{itemize}
\end{itemize}

\section{Precios}
\label{sec:precios}

\subsection{Precios de los alojamientos}
\label{sec:alojamiento}

Los precios están establecidos según del tipo de bungalow seleccionado y variarán en función de la época del año. 

Para el cálculo de los precios aplicables se ha realizado un estudio de la competencia, de manera que los precios finales resulten atractivos a los clientes.

En la Figura \ref{fig:alquiler} se muestra una tabla con toda la información relativa al alquiler de los bungalows.

\begin{figure}[h]
  \begin{center}
    \includegraphics[scale=0.6]{images/alquiler.jpg}
    \caption{Precios del alquiler}
    \label{fig:alquiler}
  \end{center}
\end{figure}

\emph{NOTA}: Se admiten mascotas en los bungalows.

Además:
\begin{itemize}
\item IVA incluido en todos los precios.
\item Se aplicará tarifa de temporada alta en los puentes.
\item Estancia mínima : 1 noche
\item Precios especiales para grupos con régimen de pensión alimenticia.
\end{itemize}

\subsection{Precios del restaurante}
\label{sec:precioRest}

\subsubsection{Para empezar}
\label{sec:carta}

En el Cuadro \ref{tab:carta} se detallan los precios en euros de cada plato ofertado en la carta del restaurante.

\begin{table}[h]
\centering
\begin{tabular}{| p{10 cm}| p{1.5cm} |}
\hline
Falafel con lechuga de la huerta\hspace{0.5cm}                           \vcenteredinclude{icon.png}                   & 3,49\euro \\ \hline
\begin{tabular}[c]{@{}c@{}}Degustación de \emph{Patés vegetarianos}\hspace{0.5cm}  \vcenteredinclude{icon.png}\\ \end{tabular} & 4,98\euro \\ \hline
\begin{tabular}[c]{@{}c@{}}Ensalada \emph{Quercus}\hspace{0.5cm}  \vcenteredinclude{icon.png}\\ \end{tabular}                    & 6,64\euro \\ \hline
\begin{tabular}[c]{@{}c@{}}Témpura de verduras\hspace{0.5cm}  \vcenteredinclude{icon.png}\\ \end{tabular}                 & 4,98\euro \\ \hline
\begin{tabular}[c]{@{}c@{}}Parrillada de verduras\hspace{0.5cm} \vcenteredinclude{icon.png}\\\end{tabular}              & 3,32\euro \\ \hline
\end{tabular}
\caption{Precios de los platos de la carta}
\label{tab:carta}
\end{table}

\emph{NOTA}: \vcenteredinclude{icon.png} Productos aptos para veganos (vegetarianos estrictos).

\newpage
\subsubsection{Primeros Platos}
\label{sec:primerosPla}
En el Cuadro \ref{tab:primerosPlatos} se detallan los precios en euros de cada uno de los primeros platos ofertados en el restaurante.
\begin{table}[h]
\centering
\begin{tabular}{|l|l|}
\hline
Falsos escalopines de tofu al pesto \hspace{0.5cm}  \vcenteredinclude{iconB.png}& 8,31\euro  \\ \hline
Risotto de setas de temporada y espárragos trigueros\hspace{0.5cm}  \vcenteredinclude{iconB.png} & 9,97\euro  \\ \hline
Patatas revolconas de seitán \hspace{0.5cm}  \vcenteredinclude{icon.png} & 9,97\euro  \\ \hline
Revuelto de setas de temporada \hspace{0.5cm}  \vcenteredinclude{iconB.png} & 11,63\euro \\ \hline
Huevo roto sobre soja texturizada adobada \hspace{0.5cm}  \vcenteredinclude{iconB.png} & 4,98\euro  \\ \hline
\end{tabular}
\caption{Precios de los primeros platos del restaurante}
\label{tab:primerosPlatos}
\end{table}

\subsubsection{Segundos Platos}
\label{sec:segundosPla}
En el Cuadro \ref{tab:segundosPlatos} se detallan los precios en euros de cada uno de los segundos platos ofertados en el restaurante.
\begin{table}[h]
\centering
\begin{tabular}{|l|l|}
\hline
Falso cordón blue de seitán con salsa de piquillo \hspace{0.5cm}  \vcenteredinclude{iconB.png}& 18,27\euro \\ \hline
Hamburguesa de lentejas con arroz \hspace{0.5cm}  \vcenteredinclude{icon.png}& 8,31\euro  \\ \hline
Hamburguesa de zanahoria\hspace{0.5cm}  \vcenteredinclude{icon.png}          & 8,31\euro  \\ \hline
Falso escalope de seitán a la mostaza con nido de patata paja y huevo mollet \vcenteredinclude{iconB.png} & 9,97\euro  \\ \hline
Moussaka \hspace{0.5cm}  \vcenteredinclude{iconB.png} & 9,97\euro  \\ \hline
Canelones de berenjena rellenos de pisto con pastel de patatas \hspace{0.5cm}  \vcenteredinclude{iconB.png}& 4,98\euro  \\ \hline
Lasaña de setas \hspace{0.5cm}  \vcenteredinclude{iconB.png}              & 14,95\euro \\ \hline
\end{tabular}
\caption{Precios de los segundos platos del restaurante}
\label{tab:segundosPlatos}
\end{table}

\newpage
\subsubsection{Postres}
\label{sec:postresPre}
En el Cuadro \ref{tab:postres} se detallan los precios en euros de cada uno de los postres ofertados en el restaurante.
\begin{table}[h]
\centering
\begin{tabular}{|l|l|}
\hline
Helado de uvas \hspace{0.5cm}  \vcenteredinclude{iconB.png}                & 3,32\euro  \\ \hline
Tiramisú\hspace{0.5cm}  \vcenteredinclude{iconB.png}                       & 10,13\euro \\ \hline
Brownie con helado de vainilla\hspace{0.5cm}  \vcenteredinclude{iconB.png} & 12,13\euro \\ \hline
Batido de frutos del bosque \hspace{0.5cm}  \vcenteredinclude{icon.png}   & 4,98\euro  \\ \hline
Batido de uvas \hspace{0.5cm}  \vcenteredinclude{icon.png}                & 3,32\euro  \\ \hline
\end{tabular}
\caption{Precios de los postres del restaurante}
\label{tab:postres}
\end{table}

Además de todos estos platos ofrecidos en la carta, el restaurante dispone de \emph{Menú del Día}, en el cuál podrán aparacer algunos platos de esta misma carta, u otros elaborados para la ocasión. 

\subsubsection{Precio medio}
\label{sec:precioMedio}
Para obtener el precio medio ofertado en el restaurante se emplea la siguiente fórmula:

\emph{Precio medio ofertado = suma de precios de venta / nº de platos contenidos en la carta}

Por lo que el precio medio ofertado es de 173,44/22 = \textbf{7,88\euro}

\subsection{Precio de la carta de vinos}
\label{sec:vinos}

En los Cuadros \ref{tab:vinosBlancos} y \ref{tab:vinosTintos} se detallan los precios en euros por botella de vino ofertado en nuestras instalaciones.

\subsubsection{Vinos Blancos}

\begin{table}[h]
\centering
\begin{tabular}{|l|l|}
\hline
Bicos (d.o. Rías Baixas)  \vcenteredinclude{iconB.png}           & 4,75\euro \\ \hline
Navesur (d.o. Rivera del Duero) \vcenteredinclude{iconB.png}     & 4,25\euro \\ \hline
Señorío Real (d.o. Rivera del Duero)\hspace{0.5cm}\vcenteredinclude{iconB.png} & 4,19\euro \\ \hline
\end{tabular}
\caption{Precios de la carta de vinos blancos}
\label{tab:vinosBlancos}
\end{table}

\newpage
\subsubsection{Vinos Tintos}

\begin{table}[h]
\centering
\begin{tabular}{|l|l|}
\hline
Camino de Castilla    \vcenteredinclude{iconB.png}                        & 8,75\euro  \\ \hline
Señorío Real (d.o. Rivera del Duero)   \vcenteredinclude{iconB.png}       & 4,50\euro  \\ \hline
Finca Nueva Reserva 2007 (d.o. Rioja) Crianza\vcenteredinclude{iconB.png} & 16,14\euro \\ \hline
Vizconde de Castilla (d.o. Toro) Crianza \vcenteredinclude{iconB.png}     & 4,65\euro  \\ \hline
\end{tabular}
\caption{Precios de la carta de vinos tintos}
\label{tab:vinosTintos}
\end{table}

\subsection{Precios de la carta de cervezas}
\label{sec:cervezas}

En el Cuadro \ref{tab:cervezas} se detallan los precios en euros por botella de 33cl de cerveza ofertada en nuestras instalaciones.

\begin{table}[h]
\centering
\begin{tabular}{|l|l|}
\hline
Cerveza Gredos    \vcenteredinclude{iconB.png}    & 4\euro \\ \hline
Burro de Sancho Rubia \hspace{0.5cm}\vcenteredinclude{iconB.png}& 4\euro \\ \hline
Sagra Premium    \vcenteredinclude{iconB.png}     & 4\euro \\ \hline
Sagra Ipa       \vcenteredinclude{iconB.png}      & 5\euro \\ \hline
\end{tabular}
\caption{Precios de la carta de cervezas}
\label{tab:cervezas}
\end{table}

\subsection{Precios de cursos de fin de semana}
\label{sec:preCursos}

\subsubsection{Curso de cocina (8 horas)}

\begin{itemize}
\item \textbf{Para dos personas}. Incluye: Alojamiento 2 días, Productos usados durante el curso (para las elaboraciones que posteriormente degustarán) y formación. 200\euro
\item \textbf{Para cuatro personas}. Incluye: Alojamiento 2 días, Productos usados durante el curso (para las elaboraciones que posteriormente degustarán) y formación. 380\euro
\item \textbf{Para grupos de 8 personas}. Incluye: alojamiento 2 días, Productos usados durante el curso ((para las elaboraciones que posteriormente degustarán) y formación. 700\euro
\end{itemize}

\subsubsection{Curso de recolección de setas, y curso de jardinería (6 horas)}
\begin{itemize}
\item \textbf{Para dos personas} .Incluye: Alojamiento 2 días y formación 100\euro
\item \textbf{Para cuatro personas}. Incluye: Alojamiento 2 días y formación 150\euro
\item \textbf{Para ocho personas}. Incluye: Alojamiento 2 días y formación 300\euro
\end{itemize}

\subsubsection{Curso de agricultura ecológica (6 horas)}
\begin{itemize}
\item \textbf{Para dos personas} .Incluye: Alojamiento 2 días y formación 100\euro
\item \textbf{Para cuatro personas}. Incluye: Alojamiento 2 días y formación 150\euro
\item \textbf{Para ocho personas}. Incluye: Alojamiento 2 días y formación 300\euro
\end{itemize}

\subsubsection{Curso de jardinería (6 horas)}
\begin{itemize}
\item \textbf{Para dos personas} .Incluye: Alojamiento 2 días y formación 100\euro
\item \textbf{Para cuatro personas}. Incluye: Alojamiento 2 días y formación 150\euro
\item \textbf{Para ocho personas}. Incluye: Alojamiento 2 días y formación 300\euro
\end{itemize}
