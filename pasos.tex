\chapter{Puesta en marcha del complejo hostelero}
\label{chap:pasos}

Pasos seguidos y a seguir para la puesta en marcha del complejo hostelero:


\section{Cuenta atrás de un restaurante}
\label{sec:cuentaAtras}

\subsection{Fase conceptual}
\label{sec:conceptual}

18 meses -12 meses.

\begin{enumerate}
\item Estudio de mercado.
\item Análisis de las necesidades.
\item Definición del producto:
	\begin{enumerate}
		\item Nombre del complejo hostelero
		\item Concepto de la restauración
	\end{enumerate}
\item Elección del público-objetivo al que se quiere llegar.
\item Contacto con los arquitectos y financieros/discusión.
\item Anteproyecto.
\item Estudio financiero.
\item El proyecto se somete a las autoridades.
\item Proyecto de ejecución.
\item Contratación del director.
\item Metodología para la búsqueda de un precio máximo( precios de venta de las prestaciones de la competencia )
\item Programa general de ventas ( carta de platos, precios de cursos, precios de bungalows, fichas técnicas, etc.)
\item Autorizaciones administrativas.
\end{enumerate}

\subsection{Fase de Organización I}
\label{sec:organizacion1}

12 meses - 6 meses

\begin{enumerate}
\item Equipamiento del complejo hostelero ( cocina, vestuario, oficina, bungalows, etc.)
	\begin{enumerate}
	\item Maquinarias.
	\item Mobiliario.
	\end{enumerate}
\item Adquisición de la ropa blanca y uniformes, pequeño inventario ( vasos, cubiertos, vajilla, material de oficina)
\item Organigrama/ cálculo de la plantilla/pliego de condiciones.
\item Definición del modo de explotación / días-horas de apertura, etc.
\item Presupuesto de explotación/ implantación de la contabilidad.
\item Dossiers de seguros y contratos de mantenimiento.
\item Impresión de los documentos administrativos.
\item Programa de contratación de los directivos ( chefs de cocina, jefes de servicio)
\item Objetivos de marketing y elección de una estrategia de comercialización.
\end{enumerate}

\subsection{Fase de Organización II}
\label{sec:organizacion2}

6 meses - 3 meses

\begin{enumerate}
\item Preparación del reglamento interno.
\item Preparación del plan de empresa.
\item Solucionar las cuestiones relativas al lavado de mantelería como de uniformes.
\item Plan de utilización de los espacios frigoríficos, reservas, etc.
\item Adquisición de los accesorios del restaurante.
\item Elección de una música de ambiente para el restaurante.
\item Control y corrección de las fichas técnicas con el chef de cocina.
\item Impresión de cartas, menús y cartas de bebidas.
\item Rótulos/ señalización interna.
\item Definición de una política medioambiental.
\end{enumerate}

\subsection{Fase de Lanzamiento}
\label{sec:lanzamiento}

3 meses -1 mes

\begin{enumerate}
\item Negociaciones con los proveedores/ elección y lista de proveedores de mercancías no perecederas y perecederas.
\item Definición de una animación de apertura.
\item Plan de comercialización para los3 meses siguientes a la apertura ( publicidad externa, merchandising interno)
\item Reserva de personal extra para la inauguración.
\item Contratación del personal. Contratación prevista para 10-15 días antes de la inauguración según los puestos de trabajo.
\item Envío por correo de las invitaciones para la inauguración.
\end{enumerate}

\subsection{Fase previa a la apertura}
\label{sec:previa}
1 mes - 10 días

\begin{enumerate}
\item Control de los puntos nos resueltos y toma de decisiones en consecuencia.
\item Petición y recepción de las mercancías no perecederas.
\item Ensayo general de todo el equipamiento con los jefes de servicio incluyendo la iluminación de emergencias, los apartados de seguridad.
\item Primera limpieza general.
\item Adjudicación y aceptación nominativas de equipamientos específicos.
\item Fijación de los precios de venta definitivos ( últimas correcciones eventuales para los platos del día cuyos precios no están impresos en las cartas)
\item Distribución de los uniformes.
\item Información a la política de cara a la inauguración.
\end{enumerate}

\subsection{Fase de Apertura}
\label{sec:apertura}

Días atrás

\begin{itemize}[label={}]
\item \textbf{X-10} Curso para todos los mandos.
\item \textbf{X-9} Curso para todo los mandos. Formación específica para los empleados ( utilización de los equipamientos, máquinas, informática, etc.).
\item \textbf{X-8} Curso de seguridad ( vandalismo, fuego, robo, accidentes)
\item \textbf{X-7} Cursos de previsión de ventas
\item \textbf{X-6} Día libre
\item \textbf{X-5} Limpieza y arreglo ( poner orden)
\item \textbf{X-4} Instrucciones sobre la inauguración oficial o apertura.
\item \textbf{X-3} Preparación de los platos de la carta. Degustación. Críticas.
\item \textbf{X-2} Puesta a punto de todos los departamentos.
\item \textbf{X-1} Puesta a punto de todos los dptos., decoración floral, dossier de prensa. Media jornada de descanso.
\item \textbf{X}   Inauguración oficial.
\item \textbf{X+1} Limpieza general. Análisis de los puntos débiles eventuales en la jornada oficial. Correcciones inmediatas si fuera necesario. Adoptar un ritmo normal de trabajo. Planificación de los platos del día. Planificación de reuniones de trabajo por sector o reuniones de mejora. Agradecimientos de la dirección a los invitados y a los empleados)
\item \textbf{X+10} Control de la formación del personal. Primer chequeo de la empresa.
\end{itemize}

\section{Horarios}
\label{sec:horarios}

\begin{figure}[h]
  \begin{center}
    \includegraphics[scale=0.5]{images/horarios.jpg}
    \caption{Horarios de atención al público}
    \label{fig:horarios}
  \end{center}
\end{figure}

\section{Fecha de los cursos}
\label{sec:cursos}

Los cursos ofrecidos por la empresa se realizaran en temporada baja, según aparece en el calendario.
Realizándose de principalmente los fines de semana:

\begin{itemize}
\item de 8:00 h a 11:00  cursos forestales
\item de 18:00 h a 21:00 cursos de cocina
\end{itemize}

\emph{Nota}: los cursos solo se realizarán bajo acuerdo previo telefónicamente.

En la Figura \ref{fig:cursos} se muestra el calendario de cursos previsto para el año 2014.

\begin{figure}[ht]
  \begin{center}
    \includegraphics[scale=0.6]{images/cursos.jpg}
    \caption{Calendario de cursos en las instalaciones}
    \label{fig:cursos}
  \end{center}
\end{figure}

\newpage
\section{A qué nos queremos parecer}
\label{sec:parecer}

El objetivo es conseguir unas instalaciones de muy alta calidad, con un excelente diseño y alto grado de funcionalidad para los empleados y comodidad para los clientes.

\begin{figure}[h]
  \begin{center}
    \includegraphics[scale=0.4]{images/parecer1.jpg}
    \caption{Fotografía detalle de la sala}
    \label{fig:parecer1}
  \end{center}
\end{figure}

\begin{figure}[ht]
  \begin{center}
    \includegraphics[scale=0.4]{images/parecer2.jpg}
    \caption{Fotografía del interior de la cocina}
    \label{fig:parecer2}
  \end{center}
\end{figure}
