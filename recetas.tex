\chapter{Recetas del restaurante}
\label{chap:recetas}

\section{Falafel con lechuga de la huerta}
\label{sec:falafel}

Se dejan los garbanzos en remojo el día anterior con agua tibia. Posteriormente se trituran hasta obtener una pasta a la cual se le incorpora la cebolla en brunoixe, el ajo en brunoixe, el impulsor, la harina, la  sal, el perejil y el comino molido. Se mezcla bien y se moldea.

Se fríen hasta que estén bien dorados.

\begin{figure}[h]
  \begin{center}
    \includegraphics[scale=0.13]{images/falafel.jpg}
    \caption{Falafel con lechuga de la huerta}
    \label{fig:falafel}
  \end{center}
\end{figure}

\section{Degustación de \emph{Patés vegetarianos}}
\label{sec:pates}

\subsection{Paté de champiñón}
\label{sec:patechampi}
\begin{itemize}
\item Picamos tanto el ajo con la cebolla
\item Doramos los ajos en aceite y luego la cebolla, hasta que esté bien pochada.
\item Añadimos al sofrito el boletus cortado en láminas y dejamos que se hagan con los otros dos ingredientes
\item Ponemos a punto de sal
\item Trituramos en la termomix hasta que nos queda una masa resultante parecida a un paté.
\end{itemize}

\subsection{Paté de berenjena}
\label{sec:pateberen}
\begin{itemize}
\item Picamos tanto el ajo con la cebolla
\item Doramos los ajos en aceite y luego la cebolla, hasta que esté bien pochada.
\item Añadimos al sofrito la berenjena troceado y dejamos que se hagan con los otros dos ingredientes
\item Ponemos a punto de sal
\item Trituramos en la termomix hasta que nos quede una masa resultante parecida a un paté.
\end{itemize}

\subsection{Humus}
\label{sec:humus}
\begin{itemize}
\item En un bol añadimos los ingredientes: garbanzos, tahini, ajo, comino, pimentón, perejil, aceite de oliva, perejil, sal y el zumo de limón. etc. A todo ello, incorporamos un poco de agua. Trituramos todos los ingredientes en la termomix.
\end{itemize}

\section{Ensalada \emph{Quercus}}
\label{sec:ensaladaQ}

\begin{figure}[h]
  \begin{center}
    \includegraphics[scale=0.12]{images/ensaladaQuercus.jpg}
    \caption{Ensalada \emph{Quercus}}
    \label{fig:ensaladaQuercus}
  \end{center}
\end{figure}

Ingredientes:
\begin{itemize}
\item Lechuga mezclum
\item Queso de cabra
\item Membrillo
\item Granada
\item Champiñón laminado
\item Aceite de oliva
\item Vinagre de Módena
\item Sal
\end{itemize}
Elaboración:
\begin{itemize}
\item Mezclar todo y aliñar con el aceite, vinagre y sal
\end{itemize}

\section{Témpura de verduras}
\label{sec:tempura}

\begin{figure}[h]
  \begin{center}
    \includegraphics[scale=0.2]{images/tempura.jpg}
    \caption{Témpura de verduras}
    \label{fig:tempura}
  \end{center}
\end{figure}

Ingredientes:
\begin{itemize}
\item Pimiento rojo en tiras
\item Pimiento verde en tiras
\item Cebolla en aros
\item Zanahoria en bastones
\item Champiñón laminado
\item Berenjena cortada a la española
\item Calabacín en rodajas.
\item Cerveza muy fría
\item Harina de trigo
\item Sal
\end{itemize}

Elaboración:
\begin{itemize}
\item Freír todos los ingredientes en la pasta resultante de mezclar la harina con la cerveza y una pizca de sal.
\end{itemize}

\section{Parrillada de verduras}
\label{sec:parrillada}

\begin{figure}[h]
  \begin{center}
    \includegraphics[scale=0.3]{images/parrillada.jpg}
    \caption{Parrillada de verduras}
    \label{fig:parrillada}
  \end{center}
\end{figure}

Ingredientes:
\begin{itemize}
\item Cebolla en rodajas.
\item Pimiento verde en tiras
\item Pimiento rojo en tiras
\item Berenjena en rodajas
\item Calabacín en rodajas
\item Champiñón en rodajas
\item Sal
\item Aceite de oliva
\end{itemize}

Elaboración:
\begin{itemize}
\item Colocar los ingredientes encima de una parrilla caliente previamente engrasada hasta que estén bien doradas las marcas de la parrilla por ambos lados.
\end{itemize}

\section{Primeros Platos}
\label{sec:primeros}

\subsection{Falsos escalopines de tofu al pesto}
\label{sec:escalopines}

Hacemos un pesto con piñones, el queso, el aceite y la albahaca.
Gran parte de ese pesto lo mezclamos con el tofu fresco escurrido y con la harina. Damos forma y enharinamos. Freír y emplatar con salsa de pesto.

\subsection{Risotto de setas de temporada y espárragos trigueros}
\label{sec:risotto}

Rehogamos con aceite el ajo, el puerro, la cebolla, los pimientos, la zanahoria, las setas y los espárragos. Añadimos el arroz y echamos el caldo de verduras poco a poco según vaya necesitando el arroz. Cuando este prácticamente al punto añadimos el queso rallado. Mezclamos bien y emplatamos.

\begin{figure}[h]
  \begin{center}
    \includegraphics[scale=0.07]{images/risotto.jpg}
    \caption{Risotto de setas de temporada y espárragos trigueros}
    \label{fig:risotto}
  \end{center}
\end{figure}

\subsection{Patatas revolconas de seitán}
\label{sec:revolconas}

\begin{figure}[h]
  \begin{center}
    \includegraphics[scale=0.7]{images/revolconas.jpg}
    \caption{Patatas revolconas de seitán}
    \label{fig:revolconas}
  \end{center}
\end{figure}

Cocemos las patatas con la cebolla. Retiramos la cebolla y hacemos un puré.
Por otro lado adobamos el seitán en trozos pequeños, los freímos con el adobo y el aceite y los retiramos. Echamos el puré de patatas y mezclamos bien. Emplatamos con algunos trozos de seitán entre el puré y unos pocos por encima.

\newpage
\subsection{Revuelto de setas de temporada}
\label{sec:revueltoTemporada}

Se hace un sofrito con el ajo picado y la cebolla, añadimos las setas y rehogamos. Echamos el huevo batido  y removemos, retirando del fuego si es preciso hasta que tengamos una textura melosa.

\begin{figure}[h]
  \begin{center}
    \includegraphics[scale=0.25]{images/revueltoTemporada.jpg}
    \caption{Revuelto de setas de temporada}
    \label{fig:revueltoTemporada}
  \end{center}
\end{figure}


\subsection{Huevo roto sobre soja texturizada adobada}
\label{sec:huevo}
\begin{itemize}
\item Hidratamos la soja texturizada, escurrimos y adobamos.
\item Freímos y ponemos un huevo mollet para que el camarero  lo corte al cliente
\end{itemize}

\section{Segundos Platos}
\label{sec:segundos}

\subsection{Falso cordón blue de seitán con salsa de piquillo}
\label{sec:cordon}

Cortamos el seitán en filetes de medio centímetro de grosor. Untamos el pate en una de las caras de los filetes y metemos una loncha de queso entre medias de las dos caras con paté. Lo empanamos con el pan rallado y los kikos. 
Para la salsa sofreímos el ajo picado y la cebolla. Añadimos el pimiento de piquillo cortado y rehogamos. Añadimos la nata y trituramos. Si resulta muy espesa pasamos por un chino.

\begin{figure}[h]
  \begin{center}
    \includegraphics[scale=0.4]{images/cordonBlue.jpg}
    \caption{Falso cordón blue de seitán con salsa de piquillo}
    \label{fig:cordonBlue}
  \end{center}
\end{figure}

\subsection{Hamburguesa de lentejas con arroz}
\label{sec:hamburguesa}

\begin{figure}[h]
  \begin{center}
    \includegraphics[scale=0.4]{images/hamburguesa.jpg}
    \caption{Hamburguesa de lentejas con arroz}
    \label{fig:hamburguesa}
  \end{center}
\end{figure}

\begin{itemize}
\item Se cubren las lentejas con agua y se cuecen a fuego lento hasta que estén blandas,
\item Se escurren y se pasan por la batidora o el pasapurés.
\item Se mezcla con las lentejas y el resto de los Ingredientes.
\item Se deja reposar todo el tiempo que se pueda para que se seque un poco la masa.
\item Se forman las hamburguesas, se ponen en una bandeja de horno engrasada y se asan en el gratinador del horno primero por un lado y luego por el otro.
\end{itemize}



\subsection{Hamburguesa de zanahoria}
\label{sec:hamburZana}

Se hace un sofrito de ajo, con la cebolla y la zanahoria. Añadimos la soja texturizada hidratada previamente y rehogamos. Añadimos harina hasta obtener una masa. Moldeamos y hacemos al horno o a la plancha. 

\begin{figure}[h]
  \begin{center}
    \includegraphics[scale=0.25]{images/hamburguesaZana.jpg}
    \caption{Hamburguesa de zanahoria}
    \label{fig:hamburguesaZana}
  \end{center}
\end{figure}

\subsection{Falso escalope de seitán a la mostaza con nido de patata paja y huevo mollet}
\label{sec:escalopeSeitan}

Para los escalopes mezclamos el gluten con todos los ingredientes hasta obtener una masa homogénea. La enrollamos con film para darle forma y lo pinchamos, para posteriormente cocerlo a fuego lento. Una  vez hecho esto, dejamos enfriar y cortamos escalopes de esta pieza de  seitán. Marcamos a la plancha.
Por otro lado cortamos las patatas en patata paja y las freímos con forma de nido. Dentro del cual ponemos un huevo mollet.

\subsection{Mousaka}
\label{sec:musaka}
\begin{itemize}
\item Se parte la cebolla fina y se rehoga en el aceite. Se añade el ajo
\item Machacado, los champiñones en láminas, los tomates pelados y cortados,
\item La soja texturizada, el caldo y el perejil y se cuece a fuego lento unos 20 minutos.
\item Se parte la berenjena en láminas finas a lo largo y se cuece al vapor.
\item En una fuente de horno untada de aceite se pone una capa de berenjenas, a continuación se cubre con una capa de la salsa. Se van poniendo capas hasta terminar con una de berenjenas. Se pone el queso rallado por encima.
\item Se baten los huevos, se mezclan con el yogurt y se pone esta mezcla por encima de las berenjenas.
\item Se mete en el horno a 1 80~ unos 30 minutos, hasta que esté dorado.
\end{itemize}

\begin{figure}[h]
  \begin{center}
    \includegraphics[scale=0.7]{images/musaka.jpg}
    \caption{Musaka}
    \label{fig:musaka}
  \end{center}
\end{figure}

\subsection{Canelones de berenjena soja con pastel de patatas}
\label{sec:canelones}
\begin{itemize}
\item Cortamos las berenjenas en lonchas muy finas con la corta fiambres y marcamos a la plancha.
\item Para el relleno hidratamos soja texturizada. Realizamos una salsa de tomate e incorporamos la soja. Dejamos cocer con la salsa. \item \item Albardamos esta farsa con las láminas de berenjenas.
\item Para el pastel cortamos las patatas en rodajas, añadimos la nata junto con los huevos batidos (previamente mezclado) junto con la sal. 
\item Horneamos hasta que el pastel esté cuajado.
\end{itemize}

\begin{figure}[h]
  \begin{center}
    \includegraphics[scale=0.2]{images/canelones.jpg}
    \caption{Canelones de berenjena soja con pastel de patatas}
    \label{fig:canelones}
  \end{center}
\end{figure}

\subsection{Lasaña de setas}
\label{sec:lasana}
\begin{itemize}
\item Por un lado cocemos las pastas de lasaña.
\item Por otro hacemos un sofrito de ajo y cebolla al que añadimos las setas y las rehogamos. 
\item Por otro hacemos una roux con el aceite de oliva y de harina tpt. Añadimos la leche caliente y varillamos hasta conseguir una bechamel sin grumos.
\item Ponemos 2 capas intercaladas de pasta y el sofrito con las setas, empezando y terminando con una de pasta y echamos la bechamel.
\end{itemize}

\begin{figure}[h]
  \begin{center}
    \includegraphics[scale=0.5]{images/lasana.jpg}
    \caption{Lasaña de setas}
    \label{fig:lasana}
  \end{center}
\end{figure}

\section{Postres}
\label{sec:postres}

\subsection{Batido de frutos del bosque}
\label{sec:batido}

Trituramos los frutos, mezclamos con el azúcar y la leche hasta que el líquido resultante quede homogéneo.

\begin{figure}[h]
  \begin{center}
    \includegraphics[scale=0.3]{images/batidoBosque.jpg}
    \caption{Batido de frutos del bosque}
    \label{fig:batidoBosque}
  \end{center}
\end{figure}

\subsection{Batido de uvas}
\label{sec:batidoUva}
Trituramos las uvas, mezclamos con el azúcar y la leche hasta que el líquido resultante quede homogéneo.

\begin{figure}[h]
  \begin{center}
    \includegraphics[scale=0.5]{images/batidoUva.jpg}
    \caption{Batido de uvas}
    \label{fig:batidoUva}
  \end{center}
\end{figure}

\subsection{Helado de uva}
\label{sec:heladoUva}

Hacemos una crema inglesa con las yemas, el azúcar y la leche. Añadimos la nata y el zumo de uva. Mezclamos todo bien y metemos en la heladera hasta que tenga la consistencia deseada.

\begin{figure}[h]
  \begin{center}
    \includegraphics[scale=0.3]{images/heladoUva.jpg}
    \caption{Helado de uva}
    \label{fig:heladoUva}
  \end{center}
\end{figure}

\subsection{Tiramisú}
\label{sec:tiramisu}

Para el bizcocho plancha:

\begin{itemize}
\item Batimos los huevos con el azúcar. Se añade la harina, el impulsor y 0,250  ml de agua templada. Emulsionamos y metemos en una manga pastelera. 
\item Extendemos en una bandeja para el horno. 
\item Horneamos a 215º durante 8 min.
\end{itemize}
Para el tiramisú:
\begin{itemize}
\item Echar la mitad del azúcar en la nata líquida y montarla hasta que quede muy compacta (debe estar fría y a ser posible en un recipiente metálico frío).
\item Separar las claras y las yemas de los huevos.
\item Montar las claras a punto de nieve (mejor si se echan unas gotas de limón o vinagre).
\item Mezclar despacio con un cucharón el queso mascarpone y las yemas de huevo, añadiendo el resto del azúcar (no utilizar la batidora).
\item Cuando el queso y las yemas estén mezclados juntar con la nata montada y las claras a punto de nieve y mezclar todo bien, pero despacio.
\item En el café caliente disolver un tercio del chocolate en polvo y echar un chorrito del..licor.
\item Mojar muy ligeramente los bizcochos en el café sin dejar que se empapen y disponer la mitad de ellos cubriendo el fondo de una fuente.
\item Echar la mitad de la mezcla sobre los bizcochos cubriéndolos uniformemente.
\item Poner una nueva capa de bizcochos mojados ligeramente sobre la mezcla.
\item Echar el resto de la mezcla sobre la segunda capa de bizcochos
\item Espolvorear bien el chocolate en polvo sobre la mezcla
\item Dejar reposar al menos 6 horas (preferible 24 horas)
\end{itemize}

\begin{figure}[h]
  \begin{center}
    \includegraphics[scale=0.25]{images/tiramisu.jpg}
    \caption{Tiramisú}
    \label{fig:tiramisu}
  \end{center}
\end{figure}

\subsection{Brownie con helado de vainilla}
\label{sec:brownieVa}

Mezclamos la harina, con la mantequilla, con la cobertura de chocolate, con los huevos, el azúcar y las nueces. Lo metemos al horno durante unos 35 min. aprox. A 180ºC el cual hemos precalentado previamente.

\begin{figure}[h]
  \begin{center}
    \includegraphics[scale=0.8]{images/brownie.jpg}
    \caption{Brownie con helado de vainilla}
    \label{fig:brownie}
  \end{center}
\end{figure}

