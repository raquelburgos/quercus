\chapter{Presentación del proyecto}
\label{chap:presentacion}

El objetivo principal es poner en marcha un complejo con bungalows y un restaurante vegetariano con productos ecológicos y de temporada, para hacer frente a la creciente demanda en el sector de \emph{clientes verdes}, denominando así, de forma escueta, a todos aquellos clientes que van  buscando cierta calidad en los productos, al no ser tratados con herbicidas ni pesticidas y por no ser de origen transgénico.

Como objetivo secundario, se pretende impartir clases de cocina a grupos de personas que quieran aprender a cocinar y cultivar estos productos por cuenta propia para llegar a ser un poco más autosuficiente.

\section{Justificación}
\label{sec:justificacion}
La idea de este negocio, surge a partir de mis dos grandes pasiones en esta vida: la cocina y el medio ambiente, y poder poner en práctica mis dos titulaciones que son Dirección de Cocina y Técnico Superior en la Gestión y Organización de los Recursos Naturales y Paisajísticos. Pudiendo así combinarlas en mi trabajo y realizarme como persona.

\section{Datos básicos de la empresa}
\label{sec:datos}
%Datos básicos: nombre, localización de la empresa

La empresa se encuentra situada en las afueras del término municipal de Poyales del Hoyo, en la provincia de Ávila.
Las coordenadas UTM de la ubicación exacta son:

X = 315.457.82 m

Y = 4.449.536.18 m

En la Figura \ref{fig:plano} se muestra dicha ubicación sobre el plano de la zona.

\begin{figure}[h]
  \begin{center}
    \includegraphics[scale=0.6]{images/plano.jpg}
    \caption{Ubicación de la empresa}
    \label{fig:plano}
  \end{center}
\end{figure}

El nombre de la empresa, \textbf{\textit{Quercus pyrenaica}}, viene a representar  una de las especies más abundantes que se encuentran a esta altura, de la familia \emph{Fagaceae}. Este género es uno de los más representativos dentro de la vegetación en la península ibérica.

La forma jurídica elegida para esta empresa es una sociedad limitada nueva empresa, cuyo nombre va a ser \emph{Jesús Burgos Plaza A2E45 SLNE}. Cuyo único socio es Jesús Burgos Plaza, con un capital mínimo de constitución de 3012 \euro. Siendo el 100\% de las acciones del único socio. 

El terreno procede de una herencia familiar, donde ya se encontraba ubicado un restaurante acondicionado con todas las instalaciones necesarias para poder continuar con la actividad. Gracias a esto, solo ha sido necesaria la compra de los bungalows para poder cambiar de categoría de restaurante a complejo hostelero. Los bungalows adquiridos son prefabricados, de manera que únicamente ha habido que instalarlos sobre el terreno. Siendo el precio de los de capacidad para cuatro personas de 50.000\euro y los de capacidad para dos de 28.000\euro.

El logotipo de la empresa es el que se muestra en la Figura \ref{fig:logo}.

\begin{figure}[htb]
  \begin{center}
    \includegraphics[scale=0.4]{logos/borador2.png}
    \caption{Logotipo de la empresa}
    \label{fig:logo}
  \end{center}
\end{figure}

\section{Identificación del socio}
\label{sec:identificacion}

Los clientes a los que va dirigido este complejo hostelero son personas que están concienciadas del medio ambiente, que quieran aprender de los cursos que se imparten desde las instalaciones o disfrutar de éstas, tanto en el restaurante como de los bungalows, además de disfrutar de las maravillosas vistas del entorno en el que está localizado el complejo y de otras actividades que se pueden realizar por la comarca.

Para poder dar los mejores servicios a nuestros clientes, la carta de presentación del socio es la que se muestra en el Cuadro \ref{tab:cartaPresentacion}.

% Please remember to add \use{multirow} to your document preamble in order to suppor multirow cells
\begin{table}[h]
\begin{tabular}{|l|lllll}
\cline{1-2}
Nombre: Jesús Burgos Plaza                                                                                                            & Edad: 29 años                                                                                                           & \multicolumn{4}{l}{\multirow{3}{*}{}} \\ \cline{1-2}
\multicolumn{2}{|l|}{\begin{tabular}[l]{@{}l@{}}Formación:\\ - Técnico Superior en la Gestión y Organización de los Recursos Naturales y Paisajísticos.\\ - Técnico Superior en Dirección de Cocina.\end{tabular}}                                                  & \multicolumn{4}{l}{}                  \\ \cline{1-2}
\multicolumn{2}{|l|}{\begin{tabular}[l]{@{}l@{}}Con experiencia en los dos sectores. Habiendo trabajado para la empresa \emph{Tragsa}\\ en varios puestos relacionados con el sector forestal y con prácticas en el restaurante\\ vegetariano \emph{Madre Tierra}.\end{tabular}} & \multicolumn{4}{l}{}                  \\ \cline{1-2}
\multicolumn{2}{l}{}                                                                                                                                                                                                                                            &         &         &         &        
\end{tabular}
\caption{Carta de presentación del socio}
\label{tab:cartaPresentacion}
\end{table}

Existe un espíritu emprendedor, ganas de llevar hacia delante el proyecto y que  permita desarrollar las capacidades adquiridas con la formación académica.

\section{Objetivos para conseguir}
\label{sec:objetivos}

\begin{itemize}
\item Facilitar a los clientes la adquisición de conocimientos forestales y agrícolas para poder ser más autosuficientes.
\item Ofrecer servicios de alojamiento y de restauración con la mayor calidad posible.
\item Fomentar la participación de nuestros mismos clientes.
\item Informar a los clientes de todas aquellas actividades que puede realizar por la zona. Para ello, varias empresas de la comarca nos hemos juntado para hacer un Stakeholder, en el cual, colaboramos entre todos, dándonos nosotros  mismos publicidad entre todas las empresas que formamos parte (Granja Escuela Casavieja, campamento parque de cuerdas, Casavieja,La Colonia de Gredos; granja escuela centro de naturaleza, Candeleda, Casa del Parque El Risquillo; centro de interpretación, Guisando; Turismo Ecuestre en el Valle del Tiétar sur de Gredos, Cabalgando en Gredos, rutas - turismo ecuestre, Arenas de San Pedro; Club Hípico Roble Alto, turismo ecuestre, Candeleda; Gredos Ecuestre, rutas - turismo ecuestre, Arenas de San Pedro; La Espuela Centro Ecuestre, La Adrada
\end{itemize}


\newpage
\section{Análisis DAFO}
\label{sec:dafo}

Mediante este análisis se pretende hacer una valoración general del proyecto, analizando las Debilidades, Amenazas, Fortalezas y Oportunidades del socio.
% Please remember to add \use{multirow} to your document preamble in order to suppor multirow cells
\begin{table}[h]
\begin{tabular}{|l|l|l|l|}
\hline
\multirow{4}{*}{\begin{tabular}[l]{@{}l@{}}Análisis\\ Interno\end{tabular}} & \multirow{4}{*}{\begin{tabular}[l]{@{}l@{}}\textbf{Debilidades:}\\ - Mi experiencia laboral en el sector \\ hostelero no es muy amplia.\end{tabular}} & \multicolumn{2}{|l|}{\multirow{4}{*}{\begin{tabular}[l]{@{}l@{}}\textbf{Fortalezas:}\\ - Poseo la formación adecuada para\\ llevar a cabo el proyecto.\end{tabular}}}  \\
                                                                            &                                                                                                                                              & \multicolumn{2}{|l|}{}                                                                                                                                        \\
                                                                            &                                                                                                                                              & \multicolumn{2}{|l|}{}                                                                                                                                        \\
                                                                            &                                                                                                                                              & \multicolumn{2}{|l|}{}                                                                                                                                        \\
                                                                            & \begin{tabular}[l]{@{}l@{}}- El hecho de que el restaurante sea\\ vegetariano limita la clientela.\end{tabular}                              & \multicolumn{2}{|l|}{\begin{tabular}[l]{@{}l@{}}- Ofertamos productos ecológicos, cuya\\ demanda está en aumento actualmente.\end{tabular}}                   \\ \hline
\begin{tabular}[c]{@{}c@{}}Análisis\\ Externo\end{tabular}                  & \begin{tabular}[l]{@{}l@{}}\textbf{Amenazas:}\\ - Creciente aumento de establecimientos\\ rurales.\end{tabular}                                       & \multicolumn{2}{|l|}{\begin{tabular}[l]{@{}l@{}}\textbf{Oportunidades:}\\ - Creciente aumento del turismo rural\\ debido a la crisis.\end{tabular}}                    \\
                                                                            &                                                                                                                                              & \multicolumn{2}{|l|}{\begin{tabular}[l]{@{}l@{}}- Incremento del turismo ecológico que\\ demanda productos y servicios como el\\ que ofertamos.\end{tabular}} \\ \hline
\end{tabular}
\caption{Análisis DAFO}
\label{tab:analisisDAFO}
\end{table}

