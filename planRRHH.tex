\chapter{Plan de recursos humanos}
\label{chap:RRHH}

\section{Organigrama}
\label{sec:organigrama}

\begin{figure}[h]
%  \begin{center}
    \includegraphics[scale=0.55]{images/organigrama.jpg}
    \caption{Organigrama de la empresa}
    \label{fig:organigrama}
%  \end{center}
\end{figure}

El servicio será realizado por personal cualificado que va a estar compuesto por:

\begin{itemize}
\item En cocina: cocinero y ayudante de cocina. Las funciones del cocinero no sólo abarcan la elaboración y servicio del producto. Sino también de la comprobación de los productos almacenados como almacenar los productos que lleguen de proveedores externos para poder tener un stockaje pequeño y evitar perdidas
\item En sala: maître y camarero. Sus funciones son las de atender a los clientes, tanto en la sala como en la terraza exterior y encargarse de tener estas dos zonas con la cubertería necesaria para realizar un servicio óptimo. El maître también se encarga de la recepción de todas las bebidas.
\item En el Office: personal de limpieza tanto para platos, cubertería como menaje utilizado en cocina.
\end{itemize}

En el Cuadro \ref{tab:sueldos} se detallan los salarios de los empleados de la empresa y las cuotas de la Seguridad Social anuales.

\begin{table}[h]
\begin{tabular}{|l|c|c|c|}
\hline
\textbf{Personal}             & \textbf{Salario Mensual} & \textbf{Seguridad Social}                                                                             & \textbf{Cantidad anual} \\ \hline
\textbf{Cocinero}             & 1000\euro                     & \begin{tabular}[c]{@{}c@{}}50\euro cada uno de los primeros 6\\ meses, los 6 restantes 125,78\euro\end{tabular} & 14.000\euro                  \\ \hline
\textbf{Ayudante de Cocina}   & 800\euro                      & \begin{tabular}[c]{@{}c@{}}50\euro cada uno de los primeros 6\\ meses, los 6 restantes 125,78\euro\end{tabular} & 11.200\euro                  \\ \hline
\textbf{Maître}               & 1000\euro                     & \begin{tabular}[c]{@{}c@{}}50\euro cada uno de los primeros 6\\ meses, los 6 restantes 125,78\euro\end{tabular} & 14.000\euro                  \\ \hline
\textbf{Camarero}             & 800\euro                      & \begin{tabular}[c]{@{}c@{}}50\euro cada uno de los primeros 6\\ meses, los 6 restantes 125,78\euro\end{tabular} & 11.200\euro                  \\ \hline
\textbf{Personal de limpieza} & 600\euro                      & \begin{tabular}[c]{@{}c@{}}50\euro cada uno de los primeros 6\\ meses, los 6 restantes 125,78\euro\end{tabular} & 9.455\euro                   \\ \hline
\textbf{Recepcionista}        & 800 \euro                     & \begin{tabular}[c]{@{}c@{}}50\euro cada uno de los primeros 6\\ meses, los 6 restantes 125,78\euro\end{tabular} & 11.200\euro                  \\ \hline
\textbf{Monitor formativo}    & 800 \euro                     & \begin{tabular}[c]{@{}c@{}}50\euro cada uno de los primeros 6\\ meses, los 6 restantes 125,78\euro\end{tabular} & 11.200\euro                  \\ \hline
\end{tabular}
\caption{Tabla de salarios del personal de la empresa}
\label{tab:sueldos}
\end{table}

En un principio el trabajo de cocinero y de monitor formativo lo piensa realizar el único socio, por lo cual, su sueldo será en función de los beneficios obtenidos por la empresa.
