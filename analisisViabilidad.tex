\chapter{Análisis de la viabilidad del proyecto }
\label{chap:viabilidad}

\begin{itemize}
\item Viabilidad de la situación: el complejo hostelero se encuentra a la falda sur de la Sierra de Gredos, en la cual, los deportistas ecológicos como pueden ser los senderistas o escaladores, frecuentan bastante esta zona y demandan productos ecológicos ya que forma parte de su estilo de vida (somos los que comemos). A éste grupo hay que sumarle los clientes que van buscando un sitio de montaña donde se encuentre rodeado de vegetación para poder pasar unos días tranquilos y a los vegetarianos que vayan buscando esa comida vegetariana que en pocos establecimientos se oferta.
\item Viabilidad técnica: al contar con todas las instalaciones y elementos necesarios para realizar todas las tareas que surgen en la empresa, considero que es viable, ya que todo el personal posee los conocimientos necesarios para llevarlo a cabo.
\item Viabilidad económica: al haber realizado una gestión y valoración de la carta del restaurante en las que ya se tiene en cuenta los gastos  fijos y variables de desempeñar todas las funciones, al igual que de todas las bebidas y de poner precios competitivos con otros establecimientos hoteleros similares, es viable.
\item Viabilidad financiera: al contar con un capital personal de 50000\euro y del terreno con casi todas las instalaciones, procedentes de una herencia familiar,  se puede hacer frente a los gastos que surjan de la empresa, ya que se posee liquidez suficiente.
\end{itemize}
